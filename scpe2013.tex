%%
%% This is file 'scpe2013.tex', 
%% An invited paper with siam macros for use with LaTeX 2e, 
%% by Paul Duggan for the Society for Industrial and Applied
%% Mathematics, October 1, 1995, Version 1.0
%% 

\documentclass[draft,a4paper]{siamltex}

\title{{ELL-i}: A scalable and inexpensive platform for fixed things}

\author{Pekka Nikander \and Vaddina Kameswar Rao\thanks{University of Turku}
  \and Hannu Tenhunen\thanks{Kungliga Tekniska H\"{o}gskolan, Stockholm}}

\begin{document}

\maketitle

\begin{abstract}
The Internet of Things (IoT) vision is enticing; each and every
``thing'' in the world is expected to be eventually connected to the
Internet.  In most of the IoT research, the focus has been in enabling
movable things to communication, including phones, tablets, RFID tags,
watches, and jewelry, to name but a few.  In such an approach, the
things are expected to have their own batteries or receive temporary
power over short distance electro-magnetic field, and the
communications is necessarily expected to take place over radio.
However, this approach has also dominated the more fixed side of the
IoT research, including a large fraction on the work on stationary
sensors and actuators, focusing also there on battery-based operations
and wireless communication. 

In this invited paper, we introduce an alternative view to the world
of stationary things.  We argue that a large majority of the fixed or
stationary things would benefit from being permanently powered using
wireline connections, and while doing so, it becomes natural to use
the same wires also for their communication needs.  We introduce the
ELL-i platform, a new open source initiative for provide a low-cost
flexible prototyping and production platform for extensible,
Power-over-Ethernet based smart appliances.  We describe the first
ELL-i prototyping board and briefly describe the planned technical
roadmap, a number of application concepts, and discuss the novel
business model. 
\end{abstract}

\begin{keywords} 
internet of things, direct current power transmission, power over
Ethernet, TBD
\end{keywords}

\pagestyle{myheadings}
\thispagestyle{plain}
\markboth{Nikander, Rao, and Tenhunen}{ELL-i}


\section{Introduction}

\begin{itemize}
  \item Refer to \cite{Atzori20102787}, define IoT
  \item Find at least two references that discuss IoT power; [1]
    discusses only RFID and battery, see their table 1. 
  \item the field: fixed appliances, lighting, actuators, sensors
  \item the problem of power: battery or powerline
  \item the ELL-i solution: PoE over ubiquitous CAT5 (or even CAT3),
    flexible, open source, inexpensive 
\end{itemize}

The rest of this paper is organised as follows.  First, in
Section~\ref{sec:related} TBD.
 
\section{Related work}
\label{sec:related}

\begin{itemize}
  \item relevant references: here I will need your group's help to gather these
  \item powering solutions in IoT
  \item any other similar PoE platforms: Arduino PoE, etc
    \begin{itemize}
    \item difference from these
    \end{itemize}
  \item PoE
    \begin{itemize}
    \item 802.3af, 802.3at, non-standard solutions to up to 65W or so
    \end{itemize}
\end{itemize}
 
\section{Platform architecture}
\label{sec:architecture}

\begin{itemize}
  \item architectural figure
  \item architecture explained
  \item current implementation
  \item plans for the next implementation round
  \item brief discussion about the design choices
\end{itemize}
 
\section{Application examples}
\label{sec:examples}

\begin{itemize}
  \item illumination: LEDs and other low-power lamps
  \item show lighting
  \item flexible retail lighting
  \item existing protocols: DALI, DMX, ...
  \item building and industrial automation
  \item description of the technical challenges here: reliability, ...
  \item education and prototyping
    \begin{itemize}
    \item brief explanation of the experiences in Grenoble
    \end{itemize}
  \item ...
\end{itemize}
 
\section{Business model}
\label{sec:business}

\begin{itemize}
  \item open source co-op
  \item contribution incentive struct
  \item licensing model
\end{itemize}
 
\section{Ongoing work}
\label{sec:ongoing}

\begin{itemize}
  \item ability to use other wiring than CAT3/CAT6
  \item RTOS
  \item flashing over Ethernet
  \item smooth path from Arduino IDE to a more advanced environment
    (Eclipse/
\end{itemize}

\section{Conclusions}
\label{sec:conclusions}

\section*{Acknowledgments}

The authors would like to TBD.

\bibliography{scpe2013}{}
\bibliographystyle{plain}

\end{document} 

