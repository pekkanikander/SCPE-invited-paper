%%
%% This is file 'scpe2013.tex', 
%% An invited paper with siam macros for use with LaTeX 2e, 
%% by Paul Duggan for the Society for Industrial and Applied
%% Mathematics, October 1, 1995, Version 1.0
%% 

\documentclass[draft,a4paper]{siamltex}

\title{{ELL-i}: A scalable and inexpensive platform for fixed things}

\author{Pekka Nikander\thanks{ELL-i open source co-operative, Helsinki}
  \and Vaddina Kameswar Rao\thanks{University of Turku}
  \and Petri Liuha\thanks{EIT ICT Labs, Helsinki Node}
  \and Hannu Tenhunen\thanks{Kungliga Tekniska H\"{o}gskolan, Stockholm}}

\begin{document}

\maketitle

\begin{abstract}
The Internet of Things (IoT) vision is enticing; each and every
``thing'' in the world is expected to be eventually connected to the
Internet.  In most of the IoT research, the focus has been in enabling
movable things to communication, including phones, tablets, RFID tags,
watches, and jewelry, to name but a few.  In such an approach, the
things are expected to have their own batteries or receive temporary
power over short distance electro-magnetic field, and the
communications is necessarily expected to take place over radio.
However, this approach has also dominated the more fixed side of the
IoT research, including a large fraction on the work on stationary
sensors and actuators, focusing also there on battery-based operations
and wireless communication. 

In this invited paper, we introduce an alternative view to the world
of stationary things.  We argue that a large majority of the fixed or
stationary things would benefit from being permanently powered using
wireline connections, and while doing so, it becomes natural to use
the same wires also for their communication needs.  We introduce the
ELL-i platform, a new open source initiative for provide a low-cost
flexible prototyping and production platform for extensible,
Power-over-Ethernet based smart appliances.  We describe the first
ELL-i prototyping board and briefly describe the planned technical
roadmap, a number of application concepts, and discuss the novel
business model. 
\end{abstract}

\begin{keywords} 
internet of things, direct current power transmission, power over
Ethernet, TBD
\end{keywords}

\pagestyle{myheadings}
\thispagestyle{plain}
\markboth{Nikander, Rao, Liuha, and Tenhunen}{ELL-i}

% ------------------------------------------------------------------

\section{Introduction}

\begin{itemize}
  \item Refer to \cite{Atzori20102787}, define IoT
  \item Find at least two references that discuss IoT power; [1]
    discusses only RFID and battery, see their table 1. 
  \item the field: fixed appliances, lighting, actuators, sensors
  \item the problem of power: battery or powerline
  \item the ELL-i solution: PoE over ubiquitous CAT5 (or even CAT3),
    flexible, open source, inexpensive 
\end{itemize}

The rest of this paper is organised as follows.  First, in
Section~\ref{sec:related} TBD.
 
% ------------------------------------------------------------------

\section{Related work}
\label{sec:related}

\begin{itemize}
  \item relevant references: here I will need your group's help to gather these
  \item powering solutions in IoT
  \item any other similar PoE platforms: Arduino PoE, etc
    \begin{itemize}
    \item difference from these
    \end{itemize}
  \item PoE
    \begin{itemize}
    \item 802.3af, 802.3at, non-standard solutions to up to 65W or so
    \end{itemize}
\end{itemize}
 
% ------------------------------------------------------------------

\section{Platform architecture}
\label{sec:architecture}

\begin{itemize}
  \item architectural figure
  \item architecture explained
  \item current implementation
  \item plans for the next implementation round
  \item brief discussion about the design choices
\end{itemize}

In this section, we describe the ELL-i technical architecture in
sufficient detail to ground the following discussion; for further
details, please refer to the open source design files at
\hbox{\tt http://www.github.com/Ell-i}.
First, we describe the overall technical architecture in a
bottom-up-fashion, including hardware, firmware, and the forthcoming
could platform.  After that, we introduce the current
implementation and the plans for the next hardware implementation.  We
conclude the section with a brief discussion of the design choices
made so far.

\subsection{Overall technical architecture}

\begin{itemize}
  \item Hardware
    \begin{itemize}
    \item Smarts -- the MCU
    \item Communications -- Ethernet
    \item Power -- Power-over-Ethernet
    \item ``Thingyness'' -- I/O subsystem
    \end{itemize}
  \item Programming environments
    \begin{itemize}
    \item Eclipse and XCode
    \item Emacs and make
    \item Compiler optimizations
    \end{itemize}
  \item Firmware -- on the board
    \begin{itemize}
    \item APIs
    \item RTOS and HAL
    \item Internet communications
    \item Optional build-in webserver
    \end{itemize}
  \item Cloud -- in the Internet
    \begin{itemize}
    \item Registration
    \item Monitoring
    \item Remote updates
    \item Community
    \end{itemize}
\end{itemize}

\subsection{Current implementation}

\begin{itemize}
  \item Hardware
    \begin{itemize}
    \item STM32F051 Cortex-M0 MCU
    \item Microchip ENC29J60 based Ethernet
    \item Linear LTXXXX based PoE
    \item Arduino compatible I/O subsystem
    \item Plastic 3D printed case
    \end{itemize}
  \item Arduino IDE
    \begin{itemize}
    \item API limitations
    \item compiler optimisations
    \end{itemize}
\end{itemize}

\subsection{Ongoing efforts}

\begin{itemize}
  \item Overall goal: Help people to kickstart from nothign to Arduino
    compatible, and then from Arduino compatible to IDE and real hardware
    development.
  \item Hardware
    \begin{itemize}
    \item 2-PCB ``sandwitch'' system; reasons why
    \item Support for 12-60V DC in
    \item Support form STM32F051 and STM32F407 (or similar)
    \item PCB support for RTC, OpAmps, level shifters, and driver FETs
    \item Mechanical enhancements
    \item Metal case
    \end{itemize}
  \item Software
    \begin{itemize}
    \item ChibiOS
    \item uIP
    \item Ibisense cloud integration
    \item Remote updates
    \end{itemize}
\end{itemize}

\subsection{Design choices}
 
\begin{itemize}
  \item Primary goal: *Low-cost* platform for PoE devices.
    \begin{itemize}
    \item Low-end ARM Cortex-M series
    \item Low-end Ethernet
    \item Connectors
    \item Mechanics
    \item ChibiOS
    \item Protocols
    \item Cloud-side
    \end{itemize}
\end{itemize}

% ------------------------------------------------------------------

\section{Application examples}
\label{sec:examples}

\subsection{Illumination}

LEDs and other low-power lamps.

\begin{itemize}
  \item show lighting
  \item flexible retail lighting
  \item existing protocols: DALI, DMX, ...
\end{itemize}

\subsection{Automation}

Building and industrial automation.

Description of the technical challenges here: reliability, ...

\subsection{Education and prototyping}

\subsubsection{EIT Smart Spaces summer school}

Brief explanation of the experiences in Grenoble.

\subsubsection{Prototyping experiences so far}

Antti's experiences.
 
% ------------------------------------------------------------------

\section{Business model}
\label{sec:business}

In this section, we briefly look at the ELL-i business model, which
somewhat deviates from most typical business models, being closer to
to the Linux kernel community~\cite{Linux-kernel-need-reference}
and XXX~\cite{Something-else-need-reference}.

As stated, the goal of the ELL-i project include providing an
inexpensive and easy-to-use platform for all kinds of fixed
appliancies.  As always, in electronics production achieving a low
price requires high volumes.  Hence, the ELL-i business model aims to
make the platform widely available at a low cost, thereby encouraging
it to be widely used, leading to increasing volumes and thereby even
lower prices.  In other words, the project does not aim to
create profits as such but to distribute the added value in the form
of lower prices, thereby contributing to the economy as a whole.

The ELL-i project itself has been incorproated as a co-operative, with
the intention that the hardware and software copyrights are co-owned
by the developers.  That is, the aim is that the co-operative owns the
copyright for all of the hardware and software components within the
core ELL-i platform and the developers who contribute to the project
may join the co-operative as members if they wish to do so.

We envision that the project and the co-op will create interest among
commercial for-profit companies.  This is important both for creating
volumes and for distributing the created value to the economy.
Hence, the ownership, governance and licencing models have been
planned and will be adjusted to make the platform lucrative for
commercial licensees.

However, in order to get the platform there in the first place, we
first need to entice a group of entusiastic developers that want to
contribute to the project.  For that, we presume that we first need to
create a level of activity close to the Arduino project, i.e. selling
a few thousand boards.  Based on studies in similar kinds of
situations~\cite{need-reference}, we expect that 1--4\% of the early
% XXX Check the percentage above
adopters will turn into active developers; see
Section~\ref{ssec:earlymarket} below.

Hence, the ELL-i business model needs to be balanced between the early
market need of creating a sizeable developer community and the
somewhat later need of attarcing commercial companies in order to
create volumes and benefitting the larger society.  We attempt to the
address these somewhat conflicting needs through separating the
ownership, governance and licencing practises; see
Sections~\ref{ssec:ownership} and~\ref{ssec:licencing} below.

The resulting structure has probably a relatively low monetary value
compared to its overall utility value, as the ownership will be more
anchored than in most alternative models.  However, we consider this
as a virtual from the macroeconomic
perspective~\cite{Olson2002}, somewhat similar to what
state-owned enterprises provide but deteched from the populistic
political system.

\subsection{Early market challenges}
\label{ssec:earlymarket}

At the moment, the main challenge ELL-i faces is a marketing one.  The
first version of the hardware platform exists both as open source
schematics and layout and as actual, functioning hardware boards.  The
software platform is being continuously enhanced, providing the basic
facilities that are needed to get started.  However, the group of
active developers is quite small, just a handful of people working
mostly on their spare time.  In order to get the ball rolling, more
active developers are needed.

We believe that the ELL-i approach of providing both power and
communications over a single cable is beneficial to a fair number of
people, especially when the incoming power can be easily converted
into a DC voltage suitable for various kinds of loads.  Hence, from
the ELL-i point of view, a major challenge is to find the people that
are already struggling with the appliance powering problem and
demonstrate the viability of the ELL-i solution.  That in turn
requires both much more visibility than what ELL-i enjoys today and a
number of additional power supply designs, making it easier to have
suitable power tailored for each load.

An initial market presense may be achieved through crowdsourcing,
e.g. through a Kickstarter~\cite{Kickstarter} campaign.  That may help
with gaining initial production volume and inducing a few new
developers; however, such a campaign is also likely to saturate the
early market, making it considerably harder to sell new hardware for a
while.  Hence, at the time a crowdsourcing campaing is commenced,
there must be sufficient structure in place so that the new developers
have a clearly laid approach and incentive to start contributing to
the project.

What comes to the powering of appliances, we basically need to learn
ourselves -- within the core group -- to build a more diverse set of
power supplies.  We also need to make the resulting knowledge
available in an form that is easier for relative newcomers to apply
than the currently available information, initially as recipies and
later on as ready-applicable DC/DC converter units.  The basic ELL-i
recipe for providing adjustable constant-current power is already
there, thereby enabling adjustable LED drivers, and we expect to
complete our first design for inductive loads, such as solenoids and
motors.

\subsection{Ownership and governance models}
\label{ssec:ownership}

ELL-i open source co-operative was founded in June and the first
bylaws were officially accepted only in this September.  In this first
set of bylaws a considerable amount of power is given to the co-op
board, trusting the initial board to have the right incentives in
establishing suitable ordinances for future governance.  At the same
time, the official ownership model is completely flat and egalitarian
in the original spirit of co-operatives, treating all members equal.
Each member may only have one membership share.

The current bylaws state that the board accepts new members.  Each new
member shall have a single membership share and they are oblidged to
pay the membership fee and a sign-up fee.

The mutual understanding within the current membership is that the
board will only accept people and businesses who have contributed to
the project.  In this early stage, though, the board may also accept
members that only have a plausible intention to contribute instead of
having already contributed.  The board may also adjust the sign-up fee
downward; e.g., in the case of members joining from developing
countries.   On the other hand, new members may opt to purchase
supplementary shares to support the co-operative.

While the ownership model is egalitarian, the governance model is
meritocratic.  That is, it aims to give more power to those members
that have contributed and are contribution most to the co-operative,
while less-contributing members have less power.  At the moment, the
bylaws state that the board divides the members into three groups
based on their contributions to the co-operative.  Those having
contributed most have 10 votes in the co-operative assembly, those in
the medium category have 3 votes, and the remaining members just one
vote.  This model is presumed to suffice in the beginning.

As the co-operative grows and the members no longer know each other at
the personal level, we expect that the governance model and maybe also
the ownership model will need revisions.  However, as there is no way
to know what model might serve the co-operative best, we assume that
the co-operative needs some organisational innovation and
experimentation.  The current presupposition is that once the number
of members passes Dunbar's number~\cite{reference-needed}, such
experimentation will become harder and more of the governance needs to
be formalised into the bylaws.

\subsection{Licencing model}
\label{ssec:licencing}

While the ownership and governance models are important for the
developers and the evolution of the co-operative structure, the
licencing model is expected to be a cornerstone of the success of the
platform.  At the moment, the exact way how the platform will be
licenced is still in the works.  It looks like that it will be
licensed with simultaneously with three different licenses, one aimed
for hobbyists, one for commercial companies that want to use the
platform in their products but that are not interested in contributing
back, and one for commercial companies that want to be part of the
community.  In the following two subsections we consider the case
first from the developers' point of view, and then from the commercial
licencees' point of view.

\subsubsection{Developer incentive structure}

Goals:
\begin{itemize}
  \item Make it easy to start with the platform, with as few
    obligations as possible
  \item Encourage individual developers to contribute back to the
    project
\end{itemize}

Structure:
\begin{itemize}
  \item Anyone may clone the source and do their own modifications
  \item Anyone wanting to contribute their modifications to the core
    must donate the IPR to the co-operative
  \item The co-operative may accept such donation or not, thereby
    creating a gradually tightening quality control barrier
  \item Anyone whose donations has been accepted may join the
    co-operative as a member
  \item Co-operative members get membership benefits and govern the
    co-operative
\end{itemize}

\subsubsection{Commercial licensee incentive structure}

Goals:
\begin{itemize}
  \item Enable commercial usage of the platform without an obligation
    of opening the kimono
  \item Encourage commercial licensees to contribute back to the
    project
\end{itemize}

Structure:
\begin{itemize}
  \item Anyone may licence the source under a GPL licence
  \item Anyone may aquire a commercial unidirectional licence from
    the co-operative
  \item Commercial licensees may join the co-operative, at the same
    time updating their license into a bidirectional one
  \item Co-operative members get membership benefits and govern the
    co-operative
\end{itemize}

\subsection{Business model summary}

% ------------------------------------------------------------------

\section{Ongoing work}
\label{sec:ongoing}

\begin{itemize}
  \item ability to use other wiring than CAT3/CAT6
  \item RTOS
  \item flashing over Ethernet
  \item smooth path from Arduino IDE to a more advanced environment
    (Eclipse/
\end{itemize}

% ------------------------------------------------------------------

\section{Conclusions}
\label{sec:conclusions}

\section*{Acknowledgments}

The authors would like to TBD.

\bibliography{scpe2013}{}
\bibliographystyle{plain}

\end{document} 

